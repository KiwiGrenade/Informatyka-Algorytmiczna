\documentclass{article}
\usepackage[top=3cm, bottom=3cm, left = 2cm, right = 2cm]{geometry}
\geometry{a4paper}
\usepackage[T1]{polski}
\usepackage[utf8]{inputenc}
\usepackage{titling}
\usepackage{caption}
\usepackage[parfill]{parskip}
\usepackage{hyperref}
\usepackage{multirow}
\usepackage{float}
\usepackage{graphicx}
\usepackage{tikz}
\usetikzlibrary{decorations.markings}
\usepackage{subcaption}
\usepackage{pgffor}

\begin{document}

\title{JFTT - Lista 4 Zadanie 6}
\author{Jakub Jaśków 268416}
\maketitle

\section*{Polecenie}
Pokaż, że język $$ L= \{a^nb^nc^i : i \neq n\} $$ nie jest bezkontekstowy.

\section*{Lematy}
\subsection*{Lemat o Pompowaniu (L.o.P.}
\textbf{Lemat o pompowaniu dla języków bezkontekstowych} - niech $L$ - język bezkontekstowy. Istnieje wtedy $n$ - (stała pompowania) zależna tylko od danego języka $L$ taka, że jeżeli $z \in L$ $\wedge$ $|z| \geq n$ to $z = uvwxy$ oraz:
\begin{itemize}
\item[•] $|vx| \geq 1$
\item[•] $|vwx| \leq n$
\item[•] $(\forall i \geq 0)$ $uv^iwx^iy \in L$
\end{itemize}
\subsection*{Lemat Ogdena}
\textbf{Lemat Ogdena} - niech $L$ - język bezkontekstowy. Instnieje wtedy $n$ - (stała pompowania) zależna od danego języka $L$ taka, że jeśli w słowie $ z \in L$ oznaczymy $\geq$ n liter to $z$ możemy zapisać jako $z = uvwxy$ oraz:
\begin{itemize}
\item[•] $v$ i $x$ mają łącznie oznaczoną co najmniej jedną literę
\item[•] $vwx$ ma co najwyżej $n$ oznaczonych liter
\item[•] $(\forall i \geq 0)$ $uv^iwx^iy \in L$
\end{itemize}
\section*{Rozwiązanie}
\subsection{L.o.P dla języków bezkontekstowych}
Najpierw sprawdźmy czy zadanie da się rozwiązać za pomocą \textbf{L.o.P dla języków bezkontekstowych}.\\\\
Załóżmy nie wprost, że $L$ - bezkontekstowy. Weźmy $z$ takie, że $z \in L$ $\wedge$ $z = a^lb^lc^k$, gdzie $l = n + 1$ $\wedge$ $l \neq k$ i $|z| \geq n$ ($n$ - stała z L.o.P).
\subsubsection*{\#1 : $k = l - 1, v = aa, x = bb$}
\begin{enumerate}
\item $i=0$, $z = a^{l-2}b^{l-2}c^k = a^{l-2}b^{l-2}c^{l-1}$
\item $i=1$, $z = a^{l}b^{l}c^k = a^{l}b^{l}c^{l-1}$
\item $i\geq2$, $z = a^{l-2(i-1)}b^{l-2(i-1)}c^k = a^{l-2(i-1)}b^{l-2(i-1)}c^{l-1}$
\end{enumerate}
\subsubsection*{\#2 : $k = l + 1, v = cc, x = \epsilon$}
\begin{enumerate}
\item $i=0$, $z = a^{l}b^{l}c^{k-2} = a^{l}b^{l}c^{l-1}$
\item $i\geq1$, $z = a^{l}b^{l}c^k = a^{l}b^{l}c^{l-1+2i}$
\end{enumerate}
\subsubsection*{\#3 : $k \leq l - 2, v = a, x = b$}
\begin{enumerate}
\item $i=0$, $z = a^{l-1}b^{l-1}c^k$, gdzie $k \leq l - 2 < l - 1$
\item $i=1$, $z = a^{l}b^{l}c^k$
\item $i\geq2$, $z = a^{l+i-1}b^{l+1-1}c^k,$ gdzie $k \leq l - 2 < l - 1 < m + i - 1$
\end{enumerate}
\subsubsection*{\#4 : $k \geq l + 2, v = c, x = \epsilon$}
\begin{enumerate}
\item $i=0$, $z = a^{l}b^{l}c^{k-1}$, gdzie $k - 1 \geq l + 1$
\item $i=1$, $z = a^{l}b^{l}c^k$
\item $i\geq2$, $z = a^{l}b^{l}c^{k+i-1},$ gdzie $k + i - 1 > l + 2$
\end{enumerate}

Niestety L.o.P. nie zadziałał - każdy rozważany wyżej przypadek należy dalej do języka.\\
\subsection*{Lemat Ogdena}
Zadanie to da się jednak rozwiązać używając \textbf{Lematu Ogdena}.\\
Załóżmy nie wprost, że $L$ - bezkontekstowy. Niech $n$ - stała z \textbf{Lematu Ogdena}. $ l = n + 1$ a $z = a^lb^lc^{l!+l}$. Musimy ozanczyć co najmniej $n$ liter, więc oznaczamy całe $b^l$. Nasze $vwx$ ma ozanczonych co najwyżej n liter. Oznacza to, że rozpatrywane przez nas części $z$ muszą zawierać:\\
\begin{enumerate}
\item przynajmniej jedno $a$ i jedno $b$,
\item tylko oznaczone $b$,
\item przynajmniej jedno $c$ i jedno oznaczone $b$,
\end{enumerate}
\textbf{Nie możemy jednocześnie pompować $a$ i $c$}.
\subsubsection*{\#1 : $vx = b^k, n\geq k \geq 1$}
$i = 0$, $a^lb^{l-k}c^{l!+l} \notin L$
\subsubsection*{\#2 : $vx = b^kc^j$, $n\leq k + j$, $k,j\geq 1$}
$i = 0$, $a^lb^{l-k}c^{l!+l-k} \notin L$
\subsubsection*{\#3 : $vx = a^kb^j$, $n\leq k + j$, $k,j\geq 1$, niech $i = l!/j + 1$}
$i = 0$, $a^{l+(l!/j)*k}b^{l+(l!/j)*j}c^{l!+l} = a^{l+(l!/j)*k}b^{l!+l}c^{l!+l}\notin L$

Z powyższych zapisów wynika, że dla wybranego słowa $z$ z języka $L$ oraz dowolnego podziału spełniającego założenia \textbf{Lematu Ogdena} znajdziemy $i$ takie, że słowo $z'=uv^iwx^iy \notin L$. Sprzeczność z założeniami -> $L$ nie jest bezkontekstowy. 

\end{document}
