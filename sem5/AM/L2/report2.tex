\documentclass{article}
\usepackage[top=3cm, bottom=3cm, left = 2cm, right = 2cm]{geometry}
\geometry{a4paper}
\usepackage[T1]{polski}
\usepackage[utf8]{inputenc}
\usepackage{titling}
\usepackage{caption}
\usepackage[parfill]{parskip}
\usepackage{hyperref}
\usepackage{multirow}
\usepackage{graphicx}
\usepackage{tikz}
\usetikzlibrary{decorations.markings}
\usepackage{subcaption}
\usepackage{pgffor}
\usepackage{float}

\begin{document}

    \title{Algorytmy metaheurystyczne - Lista 2}
    \author{Jakub Jaśków}

    \maketitle

\section*{Opis algorytmów}
\subsection*{Symulowane wyżarzanie}
W każdym kroku algorytmu wybierane jest rozwiązanie z otoczenia aktualnego rozwiązania. Jeżeli rozwiązanie to jest lepsze od aktualnego, to staje się ono aktualnym rozwiązaniem. W przeciwnym wypadku, rozwiązanie to staje się aktualnym rozwiązaniem z pewnym prawdopodobieństwem. Prawdopodobieństwo to maleje wraz z upływem czasu poprzez ustalnie aktualnej temperatury, która zmniejsza się w czasie. Zatem w początkowej fazie algorytmu prawdopodobieństwo wybrania gorszego rozwiązania jest większe, a w końcowej małe.\\\\
Początkowa temperatura ustalana jest jako $T := \alpha N$, gdzie $N$ to liczba wierzchołków.
Szukanie rozwiązania podzielone jest na epoki o ustalonej liczbie iteracji równej $S := \gamma T$. Po każdej epoce aktualna temperatura jest zmniejszana o ustalony czynnik: $T' := \beta T$.
Rozwiązanie szukane jest do momentu gdy od ustalonej liczby iteracji $M := \delta N$ nie udało się znaleźć lepszego rozwiązania.\\\\
Działanie algorytmu można opisać następująco:\\
1. Wylosuj rozwiązanie początkowe $X$.\\
2. Wybierz losowe rozwiązanie $X\prime$ znajdujące się w sąsiedztwie $X$.\\
3. Jeśli $X\prime$ jest lepsze to je przyjmij $(X := X\prime)$. W przeciwnym razie, wyznacz prawdopodobieństwo przyjęcia nowego rozwiązania używając wzoru $e(f (X)-f (X\prime))/T$ i z tym prawdopodobieństwem $X := X\prime$.\\
4. Jeśli nie wykonano jeszcze odpowiedniej liczby prób w obrębie danej epoki, wróć do punktu 2.\\
5. Zmniejsz temperaturę.\\
6. Jeśli nie osiągnięto jeszcze warunku stopu, wróć do punktu 2.\\

\subsection*{Tabu Search}
W każdym kroku algorytmu wybierane jest takie rozwiązanie z otoczenia aktualnego rozwiązania
aby było ono najlepsze spośród wszystkich rozwiązań w otoczeniu oraz nie znajdowało się na liście tabu. Jeżeli rozwiązanie to jest lepsze od aktualnego, to staje się ono aktualnym rozwiązaniem.
Każde rozwiązanie dodawane jest do listy tabu na określoną liczbę iteracji. W ten sposób algorytm może wyjść z lokalnego minimum.\\\\
Maksymalna długość listy tabu ustalana jest jako $L := \alpha N$, gdzie $N$ to liczba wierzchołków. Gdy lista tabu jest pełna, to usuwane jest z niej najstarsze rozwiązanie.
Rozwiązanie szukane jest do momentu gdy od ustalonej liczby iteracji $M := \beta N$ nie udało się znaleźć lepszego rozwiązania.\\\\
Działanie algorytmu można opisać następująco:\\
1. Wybierz rozwiązanie początkowe (np. wylosuj).\\
2. Przeszukaj otoczenie bieżącego rozwiązania (deterministycznie lub losowo) i wybierz najlepsze którego nie ma na liście.\\
3. Umieść je na liście i przyjmij jako bieżące rozwiązanie.\\
4. Sprawdź warunek stopu i jeśli nie zachodzi to wróć do punktu 2 (warunkiem stopu może być np. koniec pojemności listy, lub liczba iteracji bez znalezienia nowego najlepszego rozwiązania).\\
\section*{Poszukiwanie dogodnych parametrów}
W celu wyznaczenia najbardziej optymalnych wartości dla naszych heurystyk przeprowadzimy testy dla różnych wartości zmiennych $\alpha, \beta, \delta, \gamma$.\\\\
Wykresy poniżej zostały wyznaczone na podstawie 10 uruchomień heurystyki dla każdej kombinacji parametrów.
\subsection*{Symulowane wyżarzanie}
\subsubsection*{Temperatura początkowa - średnia długość cyklu}
    \begin{figure}[h!]
        \centering
        \includegraphics[height=9.5cm]{../../plots/sa-tuning-alpha-avg-xqf131.png}
    \end{figure}

    \begin{figure}[h!]
        \centering
        \includegraphics[height=9.5cm]{../../plots/sa-tuning-alpha-avg-xqg237.png}
    \end{figure}

\newpage

\subsubsection*{Temperatura początkowa - minimalna długość cyklu}
    \begin{figure}[h!]
        \centering
        \includegraphics[height=9.5cm]{../../plots/sa-tuning-alpha-min-xqf131.png}
    \end{figure}

    \begin{figure}[h!]
        \centering
        \includegraphics[height=9.5cm]{../../plots/sa-tuning-alpha-min-xqg237.png}
    \end{figure}

\newpage

\subsubsection*{Temperatura początkowa - średni czas działania}
    \begin{figure}[h!]
        \centering
        \includegraphics[height=9.5cm]{../../plots/sa-tuning-alpha-time-xqf131.png}
    \end{figure}

    \begin{figure}[h!]
        \centering
        \includegraphics[height=9.5cm]{../../plots/sa-tuning-alpha-time-xqg237.png}
    \end{figure}


\newpage 

\subsubsection*{Chłodzenie - średnia długość cyklu}
    \begin{figure}[h!]
        \centering
        \includegraphics[height=9.5cm]{../../plots/sa-tuning-beta-avg-xqf131.png}
    \end{figure}

    \begin{figure}[h!]
        \centering
        \includegraphics[height=9.5cm]{../../plots/sa-tuning-beta-avg-xqg237.png}
    \end{figure}

\newpage

\subsubsection*{Chłodzenie - minimalna długość cyklu}
    \begin{figure}[h!]
        \centering
        \includegraphics[height=9.5cm]{../../plots/sa-tuning-beta-min-xqf131.png}
    \end{figure}
    
    \begin{figure}[h!]
        \centering
        \includegraphics[height=9.5cm]{../../plots/sa-tuning-beta-min-xqg237.png}
    \end{figure}

\newpage

\subsubsection*{Chłodzenie - średni czas działania}
    \begin{figure}[h!]
        \centering
        \includegraphics[height=9.5cm]{../../plots/sa-tuning-beta-time-xqf131.png}
    \end{figure}
    
    \begin{figure}[h!]
        \centering
        \includegraphics[height=9.5cm]{../../plots/sa-tuning-beta-time-xqg237.png}
    \end{figure}

\newpage

\subsubsection*{Zmiana długość epoki i liczby iteracji: brak popraw średniej długości cyklu}
    \begin{figure}[h!]
        \centering
        \includegraphics[height=10.0cm]{../../plots/sa-tuning-gamma-delta-avg-xqf131.png}
    \end{figure}

    \begin{figure}[h!]
        \centering
        \includegraphics[height=10.0cm]{../../plots/sa-tuning-gamma-delta-avg-xqg237.png}
    \end{figure}

\newpage

\subsubsection*{Zmiana długość epoki i liczby iteracji: brak popraw minimalnej długości cyklu}
    \begin{figure}[h!]
        \centering
        \includegraphics[height=10.0cm]{../../plots/sa-tuning-gamma-delta-min-xqf131.png}
    \end{figure}

    \begin{figure}[h!]
        \centering
        \includegraphics[height=10.0cm]{../../plots/sa-tuning-gamma-delta-min-xqg237.png}
    \end{figure}
\newpage

\subsubsection*{Zmiana długość epoki i liczby iteracji: brak popraw średniego czasu}
    \begin{figure}[h!]
        \centering
        \includegraphics[height=10.0cm]{../../plots/sa-tuning-gamma-delta-time-xqf131.png}
    \end{figure}

    \begin{figure}[h!]
        \centering
        \includegraphics[height=10.0cm]{../../plots/sa-tuning-gamma-delta-time-xqg237.png}
    \end{figure}

\newpage

\subsection*{Tabu Search}

\subsubsection*{Zmiana długości listy i liczby iteracji: brak wpływu na średnią długość cyklu}
    \begin{figure}[h!]
        \centering
        \includegraphics[height=9cm]{../../plots/ts-tuning-alpha-beta-avg-xqf131.png}
    \end{figure}
    
    \begin{figure}[h!]
        \centering
        \includegraphics[height=9cm]{../../plots/ts-tuning-alpha-beta-avg-xqg237.png}
    \end{figure}

\newpage

\subsection*{Zmiana długości listy i liczby iteracji: brak wpływu na minimalna długość cyklu}
    \begin{figure}[h!]
        \centering
        \includegraphics[height=9cm]{../../plots/ts-tuning-alpha-beta-min-xqf131.png}
    \end{figure}
    
    \begin{figure}[h!]
        \centering
        \includegraphics[height=9cm]{../../plots/ts-tuning-alpha-beta-min-xqg237.png}
    \end{figure}

\newpage

\subsection*{Zmiana długości listy i liczby iteracji: brak wpływu na średni czas działania}
    \begin{figure}[h!]
        \centering
        \includegraphics[height=9cm]{../../plots/ts-tuning-alpha-beta-time-xqf131.png}
    \end{figure}
    
    \begin{figure}[h!]
        \centering
        \includegraphics[height=9cm]{../../plots/ts-tuning-alpha-beta-time-xqg237.png}
    \end{figure}

\newpage

\subsubsection*{Średnia długosć cyklu}
\begin{table}[h!]
    \centering
    \begin{tabular}{|c|c|c|c|c|}
        \hline 
        \multirow{2}{*}{Przykład} & \multicolumn{2}{|c|}{Losowe rozwiązania początkowe} & \multicolumn{2}{|c|}{Rozwiązanie początkowe bazujące na MST} \\
        \cline{2-5}
        \cline{2-5} & Otoczenie pełne & Otoczenie losowe & Otoczenie pełne & Otoczenie losowe \\
        \hline
        xqf131 & 610 & 640 & 602 & 620 \\
        \hline
        xqg237 & 1116 & 1166 & 1089 & 1132 \\
        \hline
        pma343 & 1487 & 1535 & 1454 & 1503 \\
        \hline
    \end{tabular}
\end{table}

\subsubsection*{Minimalna długość cyklu}
\begin{table}[h!]
    \centering
    \begin{tabular}{|c|c|c|c|c|}
        \hline 
        \multirow{2}{*}{Przykład} & \multicolumn{2}{|c|}{Losowe rozwiązania początkowe} & \multicolumn{2}{|c|}{Rozwiązanie początkowe bazujące na MST} \\
        \cline{2-5}
        \cline{2-5} & Otoczenie pełne & Otoczenie losowe & Otoczenie pełne & Otoczenie losowe \\
        \hline
        xqf131 & 582 & 593 & 594 & 594 \\
        \hline
        xqg237 & 1070 & 1118 & 1060 & 1085 \\
        \hline
        pma343 & 1433 & 1474 & 1428 & 1455 \\
        \hline
    \end{tabular}
\end{table}

\subsubsection*{Średni czas działania}
\begin{table}[h!]
    \centering
    \begin{tabular}{|c|c|c|c|c|}
        \hline 
        \multirow{2}{*}{Przykład} & \multicolumn{2}{|c|}{Losowe rozwiązania początkowe} & \multicolumn{2}{|c|}{Rozwiązanie początkowe bazujące na MST} \\
        \cline{2-5}
        \cline{2-5} & Otoczenie pełne & Otoczenie losowe & Otoczenie pełne & Otoczenie losowe \\
        \hline
        xqf131 & 0.18 & 0.007 & 0.06 & 0.004 \\
        \hline
        xqg237 & 1.9 & 0.03 & 0.5 & 0.02 \\
        \hline
        pma343 & 7.4 & 0.13 & 2.1 & 0.07 \\
        \hline
    \end{tabular}
\end{table}

\section*{Wyniki}

\begin{table}[H]
    \centering
    \begin{tabular}{|c|c||c|c||c|c||c|c|c|}
        \hline
        \multirow{2}{*}{Przykład} & \multirow{2}{*}{Optimum} & \multicolumn{2}{|c|}{Local Search} & \multicolumn{2}{|c|}{Sumulowane Wyżarzanie}  & \multicolumn{2}{|c|}{Tabu Search}  \\
        \cline{3-8}
        & & śr. waga & min. waga & śr. waga & min. waga & śr. waga & min. waga \\
        \hline\hline
        xqf131 & 564 & 620.79 & 578 & 590.25 & 567 & 604.69 & 594 \\
        \hline
        xqg237 & 1019 & 1111.17 & 1062 & 1071.57 & 1036 & 1098.06 & 1060 \\
        \hline
        pma343 & 1368 & 1452.93 & 1424 & 1369.19 & 1378 & 1455.45 & 1428 \\
        \hline
        pka379 & 1332 & 1443.34 & 1399 & 1391.45 & 1360 & 1402.36 & 1376 \\
        \hline
        bcl380 & 1621 & 1834.46 & 1728 & 1725.17 & 1685 & 1735.09 & 1707 \\
        \hline
        pbl395 & 1281 & 1432.16 & 1359 & 1335.08 & 1317 & 1337.37 & 1352 \\
        \hline
        pbk411 & 1343 & 1482.24 & 1426 & 1485.94 & 1397 & 1425.27 & 1405 \\
        \hline
        pbn423 & 1365 & 1535.68 & 1454 & 1476.04 & 1412 & 1472.93 & 1440 \\
        \hline
        pbm436 & 1443 & 1623.38 & 1535 & 1497.92 & 1481 & 1576.97 & 1535 \\
        \hline
        xql662 & 2513 & 2829.16 & 2707 & 2700.47 & 2617 & 2693.88 & 2673 \\
        \hline
        xit1083 & 3558 & 4062.82 & 3919 & 3810.48 & 3746 & 3953.64 & 3768 \\
        \hline
        icw1483 & 4416 & 5031.55 & 4839 & 4745.15 & 4628 & 4759.97 & 4733 \\
        \hline
        djc1785 & 6115 & 6792.91 & 6697 & 6527.46 & 6460 & 6471.76 & 6450 \\
        \hline
        dcb2086 & 6600 & 7516.67 & 7313 & 7137.46 & 7074 & 7172.16 & 7153 \\
        \hline
        pds2566 & 7643 & 8659.21 & 8506 & 8245.51 & 8151 & 8305.68 & 8255 \\
        \hline
    \end{tabular}
\end{table}
\section*{Parametry}
\subsection*{Symulowane wyżarzanie}
    $\alpha = 	0.5$ 	(temperatura początkowa) \\
    $\beta = 	0.95$ 	(chłodzenie)\\
    $\delta = 	0.1$ 	(max. iteracji)\\
    $\gamma = 	0.2$ 	(długość epoki)\\
    Otoczenie: pełne\\
    Rozwiązanie początkowe: losowe\\

\subsection*{Tabu Search}
    $\alpha = 0.1$ 	(długość listy tabu)\\
    $\beta = 	0.2$ 	(max. iteracji)\\
    Otoczenie: pełne\\
    Rozwiązanie początkowe: MST\\
\end{document}
