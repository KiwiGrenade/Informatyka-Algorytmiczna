\documentclass[10pt,a4paper, polish]{article}
\usepackage{lipsum}
\usepackage[normalem]{ulem}
\usepackage{array,multirow}
\usepackage[utf8]{inputenc}
\usepackage[T1]{fontenc}
\usepackage{amsfonts}
\usepackage{amssymb}
\usepackage{float}
\usepackage{graphicx}
\usepackage{latexsym,amsmath,amssymb,amsthm}
\usepackage{multicol}
\renewcommand{\arraystretch}{1,25}
\usepackage{breqn}
\usepackage{graphicx}
\usepackage{longtable} 
\usepackage{algorithm2e, algpseudocode}
\usepackage{graphicx}
\author{Jakub Jaśków}
\title{Obliczenia Naukowe - Lista nr 2}
\graphicspath{{plots/}}


\begin{document}

\maketitle

\section{zad}

\subsection{Opis}

Napisz funkcję obliczającą ilorazy różnicowe na podstawie podanych węzłów i odpowiadających im wartości funkcji bez użycia matrycy (tabeli 2D).\\

\noindent Wzór rekurencyjny dla ilorazu różnicowego k-tego rzędu:\\
1. $k = 0$  $$f[x_i] = f(x_i)$$\\
2. $k = 1$  $$f[x_i,x_j] = \frac{f(x_j) - f(x_i)}{x_j - x_i}$$  
3. $k > 1$  $$f[x_i,x_{i+1}, \ldots, x_{i+k}] = \frac{f[x_{i+1},x_{i+2},\ldots,x_{i+k}]-f[x_i,x_{i+1},\ldots,x_{i+k-1}]}{x_k-x_i}$$

\noindent Dzięki powyższemu wzorowi możemy wywnioskować, że iloraz różnicowy nie zależy od kolejności węzłów.
\subsection*{Rozwiązanie}
Dzięki znajomości węzłów $x_i$, wartości funkcji $f(x_i)$ oraz powyższego wzoru, jesteśmy w stanie stworzyć tablicę ilorazów różnicowych wyższych rzędów. Przyjmijmy $c_{ik} = f[x_i,x_{i+1}, \ldots, x_{i+k}]$. Postać tablicy:
\begin{align}
&c_{0,0} \quad c_{0,1} \quad c_{0,2} \quad \ldots \quad c_{0,k-1} \quad c_{0,k} \nonumber \\
&c_{1,0} \quad c_{1,1} \quad c_{1,2} \quad \ldots \quad c_{1,k-1} \nonumber \\
&\ldots \quad \ldots \quad \ldots \nonumber \\
&c_{k-1,0} \quad c_{k-1,1} \nonumber \\
&c_{k,0} \nonumber \\
\label{il}
\end{align}
Algorytm funkcji którą mamy stworzyć mógł by wykorzystać powyższą tablicę (gdyby nie polecenie). Nie jest to jednak rozwiązanie optymalne. Wystarczy użyć jednowymiarowej tablicy $fx$. Początkowe wartości zmiennych bierzemy z tabeli powyżej - $c_{i,0} = f(x_i)$, a następnie zamieniamy je na $c_{i-1,1}, \ldots, c_{1,i-1}, c_{0,i}$ dla każdej kolejnej zmiennej $fx_i$.\\\\
Zmienne tablicy $fx$ za każdym przejściem tworzą kolejne kolumny tablicy ilorazów różnicowych, które aktualizowane są od dołu do góry. Taka kolejność wykonywania poleceń zapewnia, że tablica $fx$ zawierać będzie ilorazy potrzebne w następnych iteracjach.

\begin{algorithm}[h]
				\DontPrintSemicolon
    			\SetKwProg{Fn}{function}{}{}
    			\SetKw{KwDownTo}{downto}
    			
			\SetKwFunction{ir}{ilorazyRoznicowe}
			\SetKwFunction{len}{length}

		    	\Fn{\ir{$x$,$f$}}{
		    		\For{$i \gets 1$ \KwTo \len{$f$}} {
		    		$fx[i] \gets f[i]$\;
		    		}
		    		\For{$i \gets 1$ \KwTo \len{$f$}} {
		    		
		    		\For{$j \gets$\len{$f$} \KwDownTo $i$} {
		    			$fx[j] \gets \dfrac{fx[j]-fx[j-1]}{x[j] - x[j-i]}$\;

		    		}
		    		}
		   
    			\KwRet $fx$\;
			}
\caption{Obliczanie ilorazów różnicowych}
\end{algorithm} 

\begin{longtable}[l]{r  c  l}
\multicolumn{1}{l}{\textbf{Dane:}}&& \\
\texttt{x}&--&wektor długości $n+1$ zawierający węzły $x_0, \ldots, x_n$, \\
\texttt{f}&--&wektor długości $n+1$ zawierający wartości interpolowanej funkcji \\
&&w węzłach $f(x_0), \ldots, f(x_n)$. \\
&& \\
\multicolumn{1}{l}{\textbf{Wyniki:}}&& \\
\texttt{fx}&--&wektor długości $n+ 1$ zawierający obliczone ilorazy różnicowe. \\
\end{longtable}

\section{zad}

\subsection*{Opis}
Napisz funkcję obliczającą wartość wielomianu interpolacyjnego stopnia $n$ w postacie Newtona $N_{n}(x)$ w punkcie $x = t$ za pomocą uogólnionego algorytmu Hornera działającej, w czasie ($O(n)$). \\

\noindent Aby lepiej zrozumieć zależność wielomianu interpolacyjnego $N_n$ od funkcji $f$ możemy przedstawić go za pomocą ilorazów różnicowych:
$$N_n(x) = \sum_{i=0}^n f[x_0,x_{1}, \ldots, x_{i}] \prod_{j=0}^{i-1}(x-x_j).$$

\noindent Zaletą powyższego wzoru jest fakt, że dodanie kolejnych $(x_i, y_i)$ nie narusza obliczonych wcześniej współczynników $c_k=f[x_0,x_1, \ldots, x_k]$.
Wartość tak przedstawionego wielomianu można łatwo obliczyć posługując się uogólnionym algorytmem Hornera.
		
		\begin{align}
		\label{horn}
			&w_n(x) := f[x_0, x_1, \ldots, x_n]& \nonumber \\
			&w_k(x) := w_{k+1}(x-x_k)+ f[x_0, x_1, \ldots, x_k]	\quad(k=n-1, n-2, \ldots, 0)& \nonumber \\
			&N_n(x) = w_0(x) \nonumber \\
		\end{align}

\subsection*{Rozwiązanie}

\begin{algorithm}[h]
				\DontPrintSemicolon
    			\SetKw{KwDownTo}{downto}
    			\SetKwProg{Fun}{function}{}{}
    			\SetKwFunction{LEN}{length}
    			\SetKwFunction{F}{warNewton}
		    	
    			\Fun{\F{$x$, $fx$, $t$}} {
		    		$n \gets \LEN{fx}$\;
		    		$nt \gets fx[n]$\;
		    		\For{$i \gets n-1$ \KwDownTo $1$} {
		    			$nt \gets fx[i] + (t - x[i]) \times nt$\; 		
		    		}
		    		\KwRet $nt$\;
    			}
    			\caption{Obliczanie wartości wielomianu interpolacyjnego w punkcie $t$}
    			\label{alg:zad2}
		\end{algorithm}			
\begin{longtable}[l]{r  c  l}
\multicolumn{1}{l}{\textbf{Dane:}}&& \\
\texttt{x}&--&wektor długości $n+1$ zawierający węzły $x_0, \ldots, x_n$, \\
\texttt{fx}&--&wektor długości $n+1$ zawierający ilorazy różnicowe \\
\texttt{t}&--&punkt, w którym należy obliczyć wartość wielomianu \\
&& \\
\multicolumn{1}{l}{\textbf{Wyniki:}}&& \\
\texttt{nt}&--&wartość wielomianu w punkcie $t$ \\
\end{longtable}


\section{zad}

\subsection*{Opis}
Znając współczynniki wielomianu interpolacyjnego w postaci Newtona $c_0 = f[x_0], c_1 = f[x_0,x_1], \ldots, c_n = f[x_0, \ldots, x_n]$ (ilorazy różnicowe) oraz węzły $x_0, x_1 \ldots, x_n$ napisać funkcję obliczającą w czasie $O(n^2)$ współczynniki jego postaci naturalnej  $a_0,\ldots,a_n$.

\subsection*{Rozwiązanie}

Aby znaleźć współczynniki wielomianu interpolacyjnego posłużymy się funkcją napisaną w poprzednim zadaniu, opartą o uogólniony algorytm Hornera.\\\\
Współczynnik $a_n$ przy najwyższej potędze $x$ jest równy $c_n$. Zatem $w_n$ z algorytmu Hornera także jest równy $a_n$. Wiedząc, że $a_n = w_n$ tworzymy wartości $a_i = a_{i+1}$. W celu znalezienia zależności pomiędzy następnymi $a_i$ przechodzimy po każdym $w_i$ ($i \in [n,\ldots,0]$) zmieniając poszczególne $a_i$ tak, że dla każdego $w_i$ w pewnym momencie dochodzimy do postaci naturalnej.

		\begin{algorithm}[h]
			\DontPrintSemicolon
%			\SetKwInOut{Input}{Input}
%    			\SetKwInOut{Output}{Output}
    			\SetKw{KwDownTo}{downto}
    			\SetKwProg{Fun}{function}{}{}
    			\SetKwFunction{LEN}{length}
    			\SetKwFunction{F}{naturalna}
		    	\Fun{\F{$x$, $fx$}} {
		    		$n \gets \LEN{fx}$\;
		    		$a[n] \gets fx[n]$\;
		    		\For{$i \gets n-1$ \KwDownTo $0$} {
		    			$a[i] \gets fx[i] - a[i+1] \times x[i]$\; 
		    			\For{$j \gets i+1$ \KwDownTo $n-1$} {
		    				$a[j] \gets a[j] - a[j+1] * x[i]$\; 	
					}	
		    		}
		    		\KwRet $a$\;
    			}

    			\caption{Współczynniki naturalne wielomianu interpolacyjnego.}
    			\label{alg:zad3}
		\end{algorithm}	
\begin{longtable}[l]{r  c  l}
\multicolumn{1}{l}{\textbf{Dane:}}&& \\
\texttt{x}&--&wektor długości $n+1$ zawierający węzły $x_0, \ldots, x_n$, \\
\texttt{fx}&--&wektor długości $n+1$ zawierający ilorazy różnicowe \\
&& \\
\multicolumn{1}{l}{\textbf{Wyniki:}}&& \\
\texttt{a}&--&wektor długości $n+1$ zawierający obliczone współczynniki postaci naturalnej \\
\end{longtable}

\section{zad}

\subsection*{Opis}
Napisać funkcję, która zinterpoluje zadaną funkcję $f(x)$ w przedziale $[a.b]$ za pomocą wielomianu interpolacyjnego stopnia n w postaci Newtona. Następnie narysuje wielomian interpolacyjny i interpolowaną funkcję. Do rysowania zainstaluj np. pakiet \textbf{Plots}, \textbf{PyPlot} lub \textbf{Gadfly}.

\subsection{Rozwiązanie}
Wydzielamy n+1 węzłów o równych odstępach pomiędzy sobą a następnie wyliczamy ich wartości funkcji. Przy użyciu węzłów, ich wartości oraz funkcji z zadania 1 wyznaczamy ilorazy różnicowe. Dzięki wykonanym wcześniej krokom i funkcji z zadania 3 możemy wyznaczyć wartości wielomianu w punkcie. Dzielimy przedział [a,b] na ciąg punktów o równych odległościach pomiędzy sobą. Dla każdego wyznaczonego punktu obliczamy wartość funkcji i wielomianu, a następnie umieszczamy na wykresie i generujemy.

\begin{longtable}[l]{r  c  l}
\multicolumn{1}{l}{Funkcja \texttt{rysujNnfx}:}&& \\
\multicolumn{1}{l}{\textbf{Dane:}}&& \\
\texttt{f}&--&zadana funkcja \\
\texttt{a, b}&--& przedział interpolacji \\
\texttt{n}&--& stopień wielomianu interpolacyjnego \\
&& \\
\multicolumn{1}{l}{\textbf{Wyniki:}}&& \\
&--&funkcja rysuje wielomian interpolacyjny i interpolowaną funkcję \\
&&w przedziale $[a,b]$ \\
\end{longtable}

\section{zad}
\subsection*{Opis}
Przetestowanie funkcji \texttt{rysujNnfx(f,a,b,n)} na przykładach:\\
$f(x) = e^x$, $[a, b] = [0,1]$, $n \in \{5,10,15\}$,\\
$f(x) = x^2\sin{x}$, $[a, b] = [-1,1]$, $n \in \{5,10,15\}$

\subsection*{Rozwiązanie}
Wywołanie funkcji \texttt{rysujNnfx(f,a,b,n)} dla danych podanych powyżej.

\subsection*{Wyniki}
		\begin{figure}[h]
			\centering
			\includegraphics[width=0.5\textwidth]{zadanie5F_5.png}
			\includegraphics[width=0.5\textwidth]{zadanie5F_10.png}
			\includegraphics[width=0.5\textwidth]{zadanie5F_15.png}
  			\caption{Wykres $e^{x}$ i jej wielomianu interpolacyjnego dla danego stopnia $n$}
  		\end{figure}		
		
		\begin{figure}[h]
			\centering
			\includegraphics[width=0.5\textwidth]{zadanie5G_5.png}
			\includegraphics[width=0.5\textwidth]{zadanie5G_10.png}
			\includegraphics[width=0.5\textwidth]{zadanie5G_15.png}
  			\caption{Wykres $x^2\sin{x}$ i jej wielomianu interpolacyjnego dla danego stopnia $n$}
  		\end{figure}	
\subsection*{Wnioski}
Obie funkcje da się bardzo łatwo i dokładnie zinterpolować co potwierdzają nakładające się na siebie wykresy
wielomianu oraz funkcji dla każdego stopnia interpolacji wielomianu.

\section{zad}

\subsection*{Opis problemu}
Przetestowanie funkcji \texttt{rysujNnfx(f,a,b,n)} (zadanie 4) na następujących przykładach:\\
$f(x) = |x|$, $[a, b] = [-1,1]$, $n \in \{5,10,15\}$,\\
$f(x) = \frac{1}{1+x^2}$, $[a, b] = [-5,5]$, $n \in \{5,10,15\}$
\subsection*{Rozwiązanie}
Wywołanie funkcji \texttt{rysujNnfx(f,a,b,n)} dla danych podanych powyżej.

\subsection*{Wyniki}
		\begin{figure}[h]
			\centering
			\includegraphics[width=0.5\textwidth]{zadanie6F_5.png}
			\includegraphics[width=0.5\textwidth]{zadanie6F_10.png}
			\includegraphics[width=0.5\textwidth]{zadanie6F_15.png}
  		\caption{Wykres $|x|$ i jej wielomianu interpolacyjnego dla danego stopnia $n$}
  		\end{figure}		
		
		\begin{figure}[h]
			\centering
			\includegraphics[width=0.5\textwidth]{zadanie6G_5.png}
			\includegraphics[width=0.5\textwidth]{zadanie6G_10.png}
			\includegraphics[width=0.5\textwidth]{zadanie6G_15.png}
  		\caption{Wykres $\frac{1}{1+x^2}$ i jej wielomianu interpolacyjnego dla danego stopnia $n$}
  		\end{figure}	
Jak możemy zaobserwować na wykresie w przeciwieństwie do poprzedniego zadania wykres funkcji oraz wielomianu stanowczo się różnią. Wzrost stopnia wielomianu nie poprawia dokładności interpolacji.\\
W sytuacji funkcji $|x|$ pojawia się kwestia nieróżniczkowalności. Można intuicyjnie stwierdzić, że wielomiany mają tendencję do bycia raczej okrągłymi, co sprawia, że znajdowanie wierzchołka wykresu staje się dla nich wyzwaniem.\\
Funkcja $f(x) = \frac{1}{1+x^2}$, $[a, b] = [-5,5]$, $n \in \{5,10,15\}$ prezentuje nam kolejne wyzwanie w postaci zjawiska Rungego - jest to zwiększanie się rozbieżności na końcach przedziału wraz ze zwiększaniem stopnia wielomianu. Można je zaobserwować w przypadku, gdy węzły interpolacji ustawione są równolegle, tak jak ma to miejsce w naszym przypadku. Jedną z metod zapobiegania takiemu obrotowi spraw jest zagęszczenie węzłów na krańcach przedziału $[a,b]$.
\subsection{Wnioski}
Kiedy znane nam są wartości funkcji tylko w niektórych punktach interpolacja wielomianowa zdaje się być dobrą metodą przybliżania funkcji. Bardzo dobrze radzi sobie z gładkimi, zaokrąglonymi funkcjami (zadanie 5). Niewłaściwe jest jednak ślepe przyjmowanie zwróconych przez interpolację wyników, ponieważ możemy się na przykład natknąć na funkcję $f(x) = |x|$ z zadanie 6, która na pierwszy rzut oka wydaje się być okej. Nie powinniśmy też na ślepo zwiększać stopnia wielomianu, ponieważ może to doprowadzić do wygenerowania jeszcze większych odstępstw od funkcji.\\
Narysowanie wykresu może dać nam wskazówki dotyczące tego jak prawidłowo powinniśmy rozmieścić węzły interpolacji.

\end{document}
